% !Mode:: "TeX:UTF-8"
\chapter{使用说明}
\section{基本范例}
\begin{table}
\begin{center}
    \begin{tabular}{c||c}
    \hline
    本科生论文基本结构 & 研究生论文基本结构\\\hline\hline
    封面 & 封面(中、英文)\\
    扉页 & 题名页、独创性声明和使用授权书\\
    中英文摘要 & 中英文摘要\\
    目录 & 目录\\
    正文 & 图表清单及主要符号表(根据情况可省略)\\
    致谢 & 主体部分\\
    参考文献 & 参考文献\\
    附录 & 附录\\
    ~~ & 攻读硕士/博士期间取得的研究\slash 学术成果\\
    ~~ & 致谢\\
    ~~ & 作者简介(仅博士生)\\
    \hline
    \end{tabular}
\end{center}
\end{table}

本科生论文结构推荐按如下的代码形式来组织整个论文。
\lstinputlisting[
    language={[LaTeX]TeX},
    caption={本科生论文结构},
    label={code-bachelor-structure},
]{sample-bachelor.tex}

研究生则推荐使用如下的代码形式来组织论文。
\lstinputlisting[
    language={[LaTeX]TeX},
    caption={研究生论文结构},
    label={code-master-structure},
]{sample-master.tex}

对于本科生或研究生的开题报告或文献综述,则推荐使用如下的代码形式组织。
\lstinputlisting[
	language={[LaTeX]TeX},
	caption={开题报告/文献综述论文结构},
	label={code-kaitireport-structure},
]{sample-kaitireport.tex}

\section{编译方法与错误排查}

本模板使用 \verb|Biber| 后端实现参考文件管理,因此编译时必须使用该命令而非 \verb|bibtex| 命令。
若计算机上没有安装适用的工具而只能使用手动编译,请使用以下工具链:\verb|xelatex -> biber -> xelatex -> xelatex|。

若使用 VSCode 与 LaTeX Workshop 进行编写,则可参考以下配置文件。
\begin{lstlisting}[caption={LaTeX Workshop 配置}]
{
    "latex-workshop.latex.recipes": [
        {
            "name": "xelatex",
            "tools": ["xelatex"]
        },
        {
            "name": "biber",
            "tools": ["biber"]
        },
        {
            "name": "Full",
            "tools": ["xelatex", "biber", "xelatex", "xelatex"]
        }
    ],
    "latex-workshop.latex.tools": [
        {
            "name": "xelatex",
            "command": "xelatex",
            "args": [
                "-synctex=1",
                "-interaction=nonstopmode",
                "-file-line-error",
                "%DOC%"
            ]
        },
        {
            "name": "biber",
            "command": "biber",
            "args": [
                "%DOCFILE%"
            ],
        }
    ]
}
\end{lstlisting}

\subsection{字体相关错误}

若编译时出现未找到字体一类错误,请检查表 \ref{tab:font-list} 中所列的字体是否安装在计算机上。
本模板使用表中带有下划线的字体族名引用字体,因此尤其注意检查这些字体与名字是否对应。
这可借助 \LaTeX{} 发行版中包含的 \verb|fc-list| 命令实现。
在 Windows 下,可于命令提示符中执行以下命令:
\begin{lstlisting}[caption={检查字体的批处理命令}]
chcp 65001 >nul & fc-list :lang=zh | find "华文楷体"
\end{lstlisting}
若命令没有产生输出,则字体未安装或未正确安装。

\begin{table}
    \centering
    \caption{使用的字体列表}
    \label{tab:font-list}
    \begin{tabular}{cc}
        \hline 
         字体族名 & 可能的字体文件名 \\ \hline
         SimSun,\underline{宋体} & simsun.ttc \\
         SimHei,\underline{黑体} & simhei.ttf \\
         STXingkai,\underline{华文行楷} & STXINGKA.TTF \\
         STKaiti,\underline{华文楷体} & STKAITI.TTF \\ \hline
    \end{tabular}
\end{table}

\section{模板选项}
\subsection{学位选项}
    \begin{itemize}
        \item bachelor---学士学位;
        \item master---硕士学位;
        \item doctor---博士学位;
        \item professional---添加该选项为专业硕士\slash 博士学位,否则为学术硕士\slash 博士学位。
        \item kaitireport---添加该选项为开题报告\slash 文献综述,否则为毕业论文。
    \end{itemize}

\subsection{其他选项}
    \begin{itemize}
        \item oneside\slash twoside---单面\slash 双面打印;
        \item openany\slash openright---新的章节在任何页面开始\slash 新的章节从奇数页开始;
        \item classfied---保密论文;
        \item color---将论文中的链接文字用颜色标识。
    \end{itemize}

\section{封面及正文前的一些设置}
\subsection{封面}
本科生论文封面直接使用\texttt{\textbackslash maketitle}命令,将编译生成论文封面和任务书(任务书中的各项需要自己在assign.tex中填写),以及“本人声明”页。只需将\texttt{data/bachelor/bachelor\_info.tex}中的信息填写完整即可自行编译生成。

研究生(包括博士研究生)的毕设论文封面使用\texttt{\textbackslash maketitle}将生成中英文封面、题名页、和独创性声明与使用授权书。只需将\texttt{data/master/master\_info.tex}中的信息填写完整即可自行编译生成。

\subsection{中英文摘要}
本科生和研究生的论文中英文摘要为\texttt{abstract.tex},请直接按照模板示例进行更改替换即可,关键词以及其他的一些个人论文信息在\texttt{data/bachelor/bachelor\_info.tex}或\texttt{data/master/master\_info.tex}中自行定义。

\subsection{目录}
生成目录为命令\texttt{\textbackslash tableofcontents},需要xelatex两遍才能正确生成目录。

对于研究生,论文还需要有图表目录以及论文主要符号表。分别使用命令\texttt{\textbackslash listoffig\hyp{}ures}和\texttt{\textbackslash listoftables},而主要符号表则在\texttt{data/master/denotation.tex}中,请自行按照模板给出的样式替换即可。

\section{正文}
\subsection{章节}
正文中的各个章节,推荐将其每一章分为单独的\texttt{.tex}文件,然后使用\texttt{\textbackslash include\{chap\hyp{}ter.tex\}}将其包含进来即可。

章节中的内容如何编写,请见\hyperref[chapter-basic]{第\ref{chapter-basic}章~~\LaTeX{}基础知识}。

\subsection{参考文献}
参考文献使用\verb|BiBLaTeX|宏包与\verb|Biber|工具,相较于较旧的\verb|BiBTeX|和\verb|natbib|能够提供更好的 Unicode 兼容与更多的功能支持。
本模板还使用了 \verb|biblatex-gb7714-2015| 宏包作为参考文件引用的样式。

基于以上选择,可按以下步骤在论文中使用和管理参考文献:
\begin{enumerate}
    \item 获取参考文献:获取 \verb|BiBLaTeX|(或\verb|BiBTeX|) 格式的参考文件条目,常见的学术搜索引擎和线上期刊均会提供该格式的导出。也可以使用 Zotero 一类的管理工具进行导出。
    \item 管理参考文献数据库:将参考文献条目存入 \verb|.bib| 拓展名的数据库文件中,并保存在主要的 \verb|.tex| 文件附近,如 \verb|data| 文件夹下,可参考本模板的 \verb|data\bibs.bib|文件。
    \item 导入参考文献数据库:使用 \verb|\bibliography| 或 \verb|\addbibresource| 宏引入参考文献数据库,该宏的第一个参数即为 \verb|.bib| 数据库文件的相对路径。具体用法可参考本模板的 \verb|sample-*.tex|文件。
    \item 引用参考文献:主要的引用方式有两种。其一是使用 \verb|\cite| 或 \verb|\upcite| 生成上标形式的引用,如 \verb|\cite{kottwitz2011latex}| 可得到这种\cite{kottwitz2011latex}引用。
        其二是使用 \verb|\parencite| 生成正文形式的引用,如 \verb|\parencite{kottwitz2011latex}| 可生成这种\parencite{kottwitz2011latex}格式的引用。
        这些宏的第一个参数须和参考文件条目中的唯一标识符(即第一个参数)一致,可参考本模板的 \verb|data\bibs.bib|文件。
    \item 输出参考文献:使用 \verb|\printbibliography| 宏输出排版好的参考文献。可参考本模板的 \verb|data\reference.tex| 文件。
\end{enumerate}

每次更新参考文献数据库后,在编译时都需要重新运行 \verb|Biber|;若未更新数据库,则不需要重新运行它。
如果最后未能生成参考文献引用或参考文献表,除了检查以上几点之外,还请检查编译命令是否正常,并尝试删除辅助文件(尤其是 \verb|.bcf| 文件)后重新编译。

\section{正文之后的内容}
\subsection{附录}
附录和正文中的章节编写方式一样。无特殊之处。
\subsection{攻读硕士\slash 博士期间所取得的研究\slash 学术成果(研究生)}
\subsection{致谢}
\subsection{作者简介(博士研究生)}
博士学位论文应该提供作者简介,主要包括:姓名、性别、出生年月日、民族、出生地;
简要学历、工作经历(职务);以及攻读学位期间获得的其它奖励(除攻读学位期间取得的研究成果之外)。
